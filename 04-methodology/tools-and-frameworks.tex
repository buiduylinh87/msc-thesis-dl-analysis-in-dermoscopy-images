\section{Tools and Frameworks}

    This section presents tools used for development and testing during the thesis. Python is selected as the main programming language for this thesis.

    \subsection{Tools}

        \paragraph{Numpy (Numerical Python)} is a scientific computing library for the Python that allows us
            to perform scientific calculations quickly \cite{oliphant2006guide}. Numpy arrays form the basis of Numpy.
            Numpy arrays are similar to python lists, but are more useful in terms of speed and functionality than python lists.

        \paragraph{Scipy} is a package for scientific computing which includes functionality several clustering algorithms,
            Fourier transforms, linear algebra, interpolation, regression, image and signal processing for the Python programming language \cite{virtanen2020scipy}.

        \paragraph{Python Imaging Library (PIL)} is a free Python library which supports several widely-used image manipulation procedures
            like per-pixed manipulating, image filtering, image enhancing, masking etc \cite{anjal2019roi}.

        \paragraph{ImageMagick} is a open-source, free image editing tool that makes many morphological operation easy
            for more than 200 image format with its built-in features like resize, flip, transform, or special filters \cite{ImageMag68online}.
            It runs on multiple thread to increase performance and supports command-line usage that makes image editing possible for scripting languages.

        \paragraph{Jupyter Notebook} is an open source web application that allows editing and running code
            which can be used with over 40 different programming languages \cite{kluyver2016jupyter}.
            It is a Json based document that has ordered cells which can be live code, equations, visualizations or narrative text.

    \subsection{Deep Learning Frameworks}

        \paragraph{TensorFlow} is an open source library for performing numerical computations. Although it can be used for computations in general,
            it is most commonly used as a tool for machine learning research.
            TensorFlow can be interfaced using Python and is then translated to a computational graph \cite{abadi2015tensorflow}.
            The computational graph can be fed with the tensors  by launching a TensorFlow session which are generalization of N-dimensional arrays.
            Weight matrices and biases are trainable variables in the TensorFlow graph during a session.
            Loss functions and optimization algorithms for backpropagation exist in TensorFlow \cite{johansen2019medical}.
            That makes training a model is as simple as specifying an objective function to optimize for, as well as running the optimizer with a batch of data inside a session.

        \paragraph{Keras} is a neural networks API for Python \cite{chollet2015}. It runs on top of TensorFlow or Theano \cite{mohan2019medical}
            which is used as the main neural network framework. Keras is user-friendly and allows for complex models to be created with relatively few lines of code.
            Keras consists of many commonly used building blocks of neural networks. These are parts as layers, objectives, activation functions and optimizers.
            The components include parts for convolutional and recurrent neural networks as convolutions, pooling, dropout and batch normalization.

        \paragraph{PyTorch} is a machine learning framework introduced by Facebook which has relatively advantages over TensorFlow
            in terms of simplicity and usability. It implements dynamic computational graphs which makes dynamic changes
            on the networks possible with a little effort. Debugging is relatively easy with Pytorch.


    \subsection{Hardware Requirements}

        \paragraph{Google Colab} which is also known as Colaboratory that requires no setup and runs entirely in the cloud is used in this thesis.
            Colab is a Jupyter Notebook environment aims to support Machine Learning and Artificial Intelligence researches for free,
            because this kind of process requires serious computational power.
            Many deep learning projects can be developed with Google Colaboratory on the default GPU processor of it, Tesla K80, using common Deep Learning frameworks and tools like Keras,
            TensorFlow and PyTorch. Google Colab runs on a connected Google Drive accounts.
            All models were trained and tested using Tesla K80 GPU which has 25 GB of video memory on a Ubuntu 18.04.

