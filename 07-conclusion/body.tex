\chapter{Conclusion}

    In skin cancers, tumors may deady and early detection can increase the survival rate.
    Advanced technologies such as deep learning are used in several fields in medicine to increase the diagnosis of the illnesses in the early stages.
    Image based analysis can help to the oncologists or the surgeons when detecting the skin tumors.
    It is clearly stated that the main purpose of the deep learning in medical imaging is to help to the medicians instead of replacing them.
    These kind of supportive methods can help to the medicians before their final diagnosis.

    In this thesis, we have addressed the problem of skin lesion segmentation, providing a unified comparision between several  state-of-the-art deep learning methods
    for medical image segmentation namely U-Net, SegAN and MultiResUNet.
    The comparision dataset is provided by ISIC for ISBI 2017 Challenge and enriched by adding Gaussian noises at different levels of sigmas.
    Insufficient data is a big challenge in medical imaging and this thesis aimed to provide accurate guidance, even if the dataset is insufficient.
    The experiment results showed that MultiResUNet and SegAN give more accurate results compared to vanilla U-Net and none of MultiResUNet or SegAN is superior to other for the all dataset.
