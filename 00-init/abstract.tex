% Keywords : English & Turkish & French, Comma seperated No more than 5 keywords
\keywords{DEEP LEARNING, IMAGE PROCESSING, MEDICAL IMAGING, MACHINE LEARNING, NEURAL NETWORKS}
\motscles{L'APPRENTISSAGE EN PROFONDEUR, TRAITEMENT D'IMAGE, L'IMAGERIE MÉDICALE, APPRENTISSAGE DE LA MACHINE, LES RÉSEAUX DE NEURONES}
\anahtarklm{DERİN ÖĞRENME, GÖRÜNTÜ İŞLEME, MEDİKAL GÖRÜNTÜLEME, MAKİNE ÖĞRENMESİ, SİNİR AĞLARI}

%
% Abstract in English
%
\abstract{Computer assisted radiology becomes an interdisciplinary domain between mathematics, medicine and engineering.
Tumor detection, analysis, classification are main problems in digital radiology for diagnosis and follow-up.
A physician or an oncologist involves in the care of patients by regarding detailed reports of carcinoma in situ that analyze the pathology of suspicious lesions.
Deep learning applied to several fields in medicine is considered as an intervention for oncology.
Even if the final treatment of the lesion is decided by the oncologists or the surgeons in a case of resection,
image based analysis of lesions (benign or malign) promises automated decision making for radiology.
Because of the increased ability of machine learning techniques to transform input data into high level presentations,
deep learning techniques used for image analysis have become important for helping physicians in the last few years.
Skin lesion detection and classification are current challenges in medical image analysis.
Skin diseases are difficult to recognize because of the similarity between lesions and low contrast between the lesions and the skin.
Dermatologic image processing benefits from the evaluation scores of neural nets. To help the physicians for the accurate diagnosis,
the medical field has an increasing interest in this technology, especially in the diagnosis of skin lesions.
This thesis presents a comparision between various state-of-the-art deep learning frameworks namely
SegAN and MultiResUNet to solve skin lesion analysis problems using a dermatoscopic image that contains a skin tumor.
SegAN which is a special type of Generative Adversial Network (GAN) brings a new architecture in machine learning by adding generator and discriminator steps in data analysis.
MultiResUNet is a U-Net-based neural network architecture which aims to overcome the insufficient data problem in medical imaging field by extracting contexual details even if the dataset is small.
In this article, SegAN and MultiResUNet architectures have been implemented on two dimensional skin lesion images from the
International Skin Imaging Collaboration (ISIC) 2017 Challenge. After the preprocessing, colored images have been trained in both SegAN and MultiResUNet.
The experiment setup has been enriched by adding incremental noise on tumor images before models training.
The evaluation has been tested through accuracy, sensitivity, specificity, Dice coefficient and Jaccard coefficient parameters.
In conclusion, test results showed that both SegAN and MultiResUNet architectures provide a robust approach in skin lesion analysis and there is no actual superiority againist each other.}

%
% Abstract in French
\resume{La radiologie assistée par ordinateur devient un domaine interdisciplinaire entre les mathématiques, la médecine et l'ingénierie.
La détection, l'analyse, la classification des tumeurs sont les principaux problèmes en radiologie numérique pour le diagnostic et le suivi.
Un médecin ou un oncologue intervient dans la prise en charge des patients en consultant des rapports détaillés de carcinome in situ qui analysent la pathologie des lésions suspectes.
L'apprentissage profond appliqué à plusieurs domaines de la médecine est considéré comme une intervention en oncologie.
Même si le traitement final de la lésion est décidé par les oncologues ou les chirurgiens en cas de résection,
l'analyse des lésions (bénignes ou malignes) basée sur l'image promet une prise de décision automatisée pour la radiologie.
En raison de la capacité accrue des techniques d'apprentissage automatique à transformer les données d'entrée en présentations de haut niveau,
les techniques d'apprentissage en profondeur utilisées pour l'analyse d'images sont devenues importantes pour aider les médecins au cours des dernières années.
La détection et la classification des lésions cutanées sont des défis actuels dans l'analyse d'images médicales.
Les maladies de la peau sont difficiles à reconnaître en raison de la similitude entre les lésions et du faible contraste entre les lésions et la peau.
Le traitement d'image dermatologique bénéficie des scores d'évaluation des réseaux neuronaux. Pour aider les médecins à poser un diagnostic précis,
le domaine médical s'intéresse de plus en plus à cette technologie, notamment au diagnostic des lésions cutanées.
Cet article présente une comparaison entre diverses approches basées sur l’apprentissage en profondeur
SegAN et MultiResUNet pour résoudre les problèmes d'analyse des lésions cutanées à l'aide d'une image dermatoscopique contenant une tumeur cutanée.
Les réseaux adverses gaussiens (SegAN) apportent une nouvelle architecture dans l'apprentissage automatique en ajoutant des étapes de générateur et de discriminateur dans l'analyse des données.
MultiResUNet est une architecture de réseau neuronal qui vise à surmonter le problème de données insuffisantes dans le domaine de l'imagerie médicale
en extrayant des détails contextuels même si l'ensemble de données est petit.
Dans cet article, les architectures SegAN et MultiResUNet ont été implémentées sur des images de lésions cutanées bidimensionnelles
Défi 2017 de la Collaboration internationale en imagerie cutanée (CITI). Après le prétraitement, des images en couleur ont été formées à la fois dans SegAN et MultiResUNet.
La configuration de l'expérience a été enrichie par l'ajout de bruit incrémentiel sur les images tumorales avant la formation des modèles.
L'évaluation a été testée par les paramètres d'exactitude, de sensibilité, de spécificité, de coefficient de dés et de coefficient de Jaccard.
En conclusion, les résultats des tests ont montré que les architectures SegAN et MultiResUNet fournissent une approche robuste dans l'analyse des lésions cutanées et qu'il n'y a pas de supériorité réelle à nouveau.}
% Turkish Abstract
%
\oz{Bilgisayar destekli radyoloji, matematik, tıp ve mühendislik arasında disiplinlerarası bir alan haline gelmiştir.
Tümör tespiti, analizi ve sınıflandırması, tanı ve takip için dijital radyolojideki temel problemlerdendir.
Bir doktor veya onkolog, şüpheli lezyonların patolojisini analiz eden in situ karsinomun detaylı raporlarını dikkate alarak hastaların bakımında yer alır.
Tıpta çeşitli alanlara uygulanan derin öğrenme, onkolojiye bir müdahale olarak kabul edilir.
Rezeksiyon durumunda lezyonun son tedavisine onkologlar veya cerrahlar karar verse bile, görüntü temelli lezyon (iyi huylu veya kötü huylu) analizi, radyolojide otomatik karar vermeyi vaat eder.
Son yıllarda makine öğrenimi tekniklerinin insan algısı için anlamsız olabilecek verileri anlamlı hale dönüştürme yeteneğinin artmasıyla beraber,
görüntü analizi için kullanılan derin öğrenme teknikleri, son yıllarda hekimlere yardımcı olan önemli bir araç haline gelmiştir.
Deri lezyonlarının saptanması ve sınıflandırılması tıbbi görüntü analizinde güncel zorluklardandır.
Lezyonlar arasındaki benzerlik ve lezyonlar ile cilt arasındaki düşük kontrast nedeniyle cilt hastalıklarını tanımak zordur.
Dermatolojik görüntü işleme, doktorların doğru tanı koymasına yardımcı olmak için bu teknolojiyle, özellikle cilt lezyonlarının teşhisiyle ilgili olan kısmına yoğun ilgi göstermektedir.
Bu makale,  Kapsül Ağı, bir generatif adversial ağ (GAN) türevi olan SegAN ve MultiResUNet gibi son teknoloji derin öğrenme yaklaşımlarıyla cilt tümörü içeren dermatoskopik
görüntüleri kullanarak cilt lezyonu analiz problemlerini çözmede yol göstermeyi amaçlayan bir karşılaştırma sunmaktadır.
SegAN, klasik CNN'lerden farklı olarak derin öğrenme modeline üretici ve ayırıcı adımlar ekleyerek yeni bir mimari sunmuştur.
MultiResUNet, medikal görüntüleme alanının temel sorunlarından biri olan yetersiz veri problemini
bağlamsal detayları az veriden başarılı bir şekilde çıkararak aşmayı amaçlayan bir derin sinir ağı mimarisidir.
Bu makalede, SegAN ve MultiResUNet mimarileri Uluslararası Cilt Görüntüleme İşbirliği (ISIC) 2017 Yarışması'ndan alınan iki boyutlu cilt lezyonu görüntüleri üzerine uygulanmıştır.
Ön işlemeden sonra, renkli görüntüler hem SegAN da hem de MultiResUNet'te eğitilmiştir. Deney düzeneği, model eğitimi öncesinde tümör görüntülerine artımlı gürültü eklenerek zenginleştirilmiştir.
Değerlendirme; doğruluk, duyarlılık, özgüllük, Dice katsayısı ve Jaccard katsayısı parametreleri ile yapılmıştır.
Sonuç olarak, test sonuçları hem SegAN hem de MultiResUNet mimarilerinin cilt lezyonu analizinde tutarlı bir yaklaşım sağladığını
ve farklı gürültü oranlarında birbirlerine karşı gerçek bir üstünlük kuramadıklarını göstermiştir.}
