\chapter{Conclusion}
\thispagestyle{empty}

Skin lesions or tumors may have severe results in human heath.
The early analysis of potential moles can increase the survival rate by using appropriate detection paradigms.
Advanced technologies such as deep learning are actually used in several fields in medicine to increase the diagnosis of the illnesses in the early stages.
Image based analysis can help to the oncologists or the surgeons when detecting the skin tumors.
It is clearly stated that the main purpose of the deep learning in medical imaging is to help to the medicians instead of replacing them.
These kind of supportive methods can help to the medicians before their final diagnosis.

In this thesis, the problem of skin lesion segmentation is addressed by providing a unified hierarchy to compare several deep learning methods.
Medical image segmentation has been performed using U-Net, SegAN and MultiResUNet.
The dataset has created along ISIC for ISBI 2017 Challenge and has been enriched by adding Gaussian noises at different levels of sigmas.
Insufficient data is a big challenge in medical imaging and this thesis aimed to provide accurate guidance, even if the dataset is insufficient.
The experiment results showed that MultiResUNet and SegAN give more accurate results compared to vanilla U-Net and none of MultiResUNet or SegAN is superior to other for the all dataset.
