%! Author = fatihergin
%! Date = 2020-05-03

\begin{longtable}{c|cccc}
\caption{Summary of related skin lesion segmentation surveys} \\
Publication                       & Architecture & Title                                                                                                                                                 & Highlights \\
\specialrule{2pt}{1pt}{1pt}
\citet{long2015fully}             & FCN          & \begin{multilinetable}Fully convolutional networks for semantic segmentation\end{multilinetable}                                                      & \begin{multilinetable}The first FCN implementation for semantic segmentation\end{multilinetable}   \\
\specialrule{0.5pt}{1pt}{1pt}
\citet{ronneberger2015u}          & U-Net        & \begin{multilinetable}U-net: Convolutional networks for biomedical image segmentation\end{multilinetable}                                             & \begin{multilinetable}A new architecture focused on medical image segmentation\end{multilinetable}   \\
\specialrule{0.5pt}{1pt}{1pt}
\citet{yuan2017automatic}         & FCN          & \begin{multilinetable}Automatic skin lesion segmentation using deep fully convolutional networks with jaccard distance\end{multilinetable}            & \begin{multilinetable}Jaccard distance bases loss function\end{multilinetable}   \\
\specialrule{0.5pt}{1pt}{1pt}
\citet{yuan2017automatic2}        & CDNN         & \begin{multilinetable}Automatic skin lesion segmentation with fully convolutional-deconvolutional networks\end{multilinetable}                        & \begin{multilinetable}Adding batch normalization to the output of CD layers\end{multilinetable}   \\
\specialrule{0.5pt}{1pt}{1pt}
\citet{yuan2017improving}         & CDNN         & \begin{multilinetable}Improving dermoscopic image segmentation with enhanced convolutional-deconvolutional networks\end{multilinetable}               & \begin{multilinetable}Discriminant capacity is optimized with smaller kernels\end{multilinetable}   \\
\specialrule{0.5pt}{1pt}{1pt}
\citet{bi2017dermoscopic}         & FCN          & \begin{multilinetable}Dermoscopic image segmentation via multistage fully convolutional networks\end{multilinetable}                                  & \begin{multilinetable}Multistage FCN with localized responsibilities\end{multilinetable}   \\
\specialrule{0.5pt}{1pt}{1pt}
\citet{yu2018melanoma}            & FCRN         & \begin{multilinetable}Melanoma recognition in dermoscopy images via aggregated deep convolutional features\end{multilinetable}                        & \begin{multilinetable}A pretrained dataset is used to extract deep representations of images\end{multilinetable}   \\
\specialrule{0.5pt}{1pt}{1pt}
\citet{al2018skin}                & FrCN         & \begin{multilinetable}Skin lesion segmentation in dermoscopy images via deep full resolution convolutional networks\end{multilinetable}               & \begin{multilinetable}Eliminates subsampling layers and learns the full resolution features directly\end{multilinetable}   \\
\specialrule{0.5pt}{1pt}{1pt}
\citet{li2018dense}               & DDN          & \begin{multilinetable}Dense deconvolutional network for skin lesion segmentation\end{multilinetable}                                                  & \begin{multilinetable}Residual learning based 3 layered network\end{multilinetable}   \\
\specialrule{0.5pt}{1pt}{1pt}
\citet{xue2018adversarial}        & SeGAN        & \begin{multilinetable}Adversarial learning with multi-scale loss for skin lesion segmentation\end{multilinetable}                                     & \begin{multilinetable}GAN based network with multi-scale loss function\end{multilinetable}   \\
\specialrule{0.5pt}{1pt}{1pt}
\citet{peng2019segmentation}      & GAN          & \begin{multilinetable}Segmentation of dermoscopy image using adversarial networks\end{multilinetable}                                                 & \begin{multilinetable}Consist of A CNN based discrimination and a U-Net based segmentation networks\end{multilinetable}   \\
\specialrule{0.5pt}{1pt}{1pt}
\citet{tu2019segmentation}        & GAN          & \begin{multilinetable}Segmentation of Lesion in Dermoscopy Images Using Dense-Residual Network with Adversarial Learning\end{multilinetable}          & \begin{multilinetable}Dense-Residual Network based segmentation block\end{multilinetable}   \\
\specialrule{0.5pt}{1pt}{1pt}
\citet{tschandl2019domain}        & FCN          & \begin{multilinetable}Domain-specific classification-pretrained fully convolutional network encoders for skin lesion segmentation\end{multilinetable} & \begin{multilinetable}Encoding layers are fed with pretrained weight\end{multilinetable}   \\
\specialrule{0.5pt}{1pt}{1pt}
\citet{ninh2019skin}              & SegNet       & \begin{multilinetable}Skin Lesion Segmentation Based on Modification of SegNet Neural Networks\end{multilinetable}                                    & \begin{multilinetable}Reduces the training parameters while keeps the accuracy\end{multilinetable}   \\
\specialrule{0.5pt}{1pt}{1pt}
\citet{mirikharaji2019learning}   & CNN          & \begin{multilinetable}Learning to segment skin lesions from noisy annotations\end{multilinetable}                                                     & \begin{multilinetable}Reliable and unreliable annotation sets are used together\end{multilinetable}   \\
\specialrule{0.5pt}{1pt}{1pt}
\citet{sarker2019mobilegan}       & MobileGAN    & \begin{multilinetable}MobileGAN: Skin Lesion Segmentation Using a Lightweight GAN\end{multilinetable}                                                 & \begin{multilinetable}Reduces the training parameters while keeps the accuracy\end{multilinetable}   \\
\specialrule{0.5pt}{1pt}{1pt}
\citet{lei2020skin}               & GAN          & \begin{multilinetable}Skin Lesion Segmentation via Generative Adversarial Networks with Dual Discriminators\end{multilinetable}                       & \begin{multilinetable}Dual discriminator module are used for discrimination block\end{multilinetable}   \\
\specialrule{0.5pt}{1pt}{1pt}
\citet{zafar2020skin}             & Res-Unet     & \begin{multilinetable}Skin Lesion Segmentation from Dermoscopic Images Using Convolutional Neural Network\end{multilinetable}                         & \begin{multilinetable}Combination of U-Net and ResNet\end{multilinetable}   \\
\specialrule{0.5pt}{1pt}{1pt}
\citet{xie2020mutual}             & MB-DCNN      & \begin{multilinetable}A Mutual Bootstrapping Model for Automated Skin Lesion Segmentation and Classification\end{multilinetable}                      & \begin{multilinetable}Consists of three sub CNNs with different responsibilities\end{multilinetable}   \\
\hline
\end{longtable}
\label{table:summary-of-related-skin-lesion-segmentation-surveys}

