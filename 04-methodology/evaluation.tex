\section{Evaluation Metrics}

    Several evaluation metrics were used to determine the quality of the models.
    Dice coefficient, Jaccard index, Accuracy, Sensitivity and Specificity  were used to compare the target and predicted segmentation mask.
    The true positives (TP) determine pixels (or voxels) correctly classified as being part of the segmentation,
    a false positive (FP) is a pixel incorrectly classified as being part of the segmentation, and a false negative (FN)
    is a pixel which should have been part of the segmentation but was not.

    \begin{itemize}

        \item \textbf{Dice coefficient} which is also known as similarity coefficient or F1 score is a similarity metrics
                computed by comparing the pixel-wise agreement between the groundtruth and its predicted segmentation.
                Specially, this metric is just used to evaluate the segmentation performance of the model.

                \begin{equation}
                    Dice &= \frac{2 * TP}{2 * TP + FN + FP} \\
                \end{equation}

        \item \textbf{Jaccard index} also known as the Jaccard similarity coefficient, is a similarity metrics
                which compares predictions with the ground truths by dividing the size of the intersections by size of the unions.

                \begin{equation}
                    Jaccard &= \frac{TP}{TP + FN + FP} \\
                \end{equation}

        \item \textbf{Accuracy} measures the proportion of true positives and true negatives whose are correctly segmented instances
                to the total number of instances. It is derived from sensitivity and specificity which are given below.

                \begin{equation}
                    Accuracy &= \frac{TP + TN}{TP + FP + TN + FN} \\
                \end{equation}

        \item \textbf{Sensitivity and Specificity} are the other metrics used in this project. Sensitivity aims to measure
            correctly segmented instance ratio while specificity measures incorrectly segmented instance ratio.

                \begin{equation}
                    Sensitivity &= \frac{TP}{TP + FN} \\
                \end{equation}

                \begin{equation}
                    Specificity &= \frac{TN}{TN + FP} \\
                \end{equation}

    \end{itemize}


